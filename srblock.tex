\documentclass{article}

% Basic packages
\usepackage{amsmath}
\usepackage{amsthm}
\usepackage{amssymb}
\usepackage{geometry}
\usepackage{tcolorbox}

% Page setup
\geometry{a4paper, margin=1in}

% Theorem environments
\newtheorem{theorem}{Theorem}
\newtheorem{definition}{Definition}
\newtheorem{exercise}{Exercise}

\begin{document}

\title{Lecture 7: Special Relativity and Blockchain Consensus}
\author{Course Notes}
\date{\today}
\maketitle

\section{Introduction}
This lecture explores the intersection of special relativity and blockchain consensus mechanisms.

\section{Properties of Local Consensus}
\begin{theorem}
In a spacetime grid governed by special relativity, consensus formation must satisfy locality conditions.
\end{theorem}

\begin{definition}
The light cone $C(p)$ of event $p$ consists of all events $q$ satisfying $\eta(q-p,q-p)=0$.
\end{definition}

\section{Mathematical Implications}
The relationship between light cones and blockchain consensus can be expressed through:
\[ S[\Phi] = \sum_{(p_i \to p_j)} f(\psi_i, \psi_j) \]

\end{document}
